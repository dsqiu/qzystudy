\documentclass{report}
\title{\bfseries 	Aesop Fables}
\author{Aesop\thanks{Thanks to the reader.}
			 \and Nobody\thanks{Thanks to nobody.}}
\date{\today}
\begin{document}
\maketitle
\begin{abstract}
The tale, the Parable, and the Fable are all common and popular
modes of conveying instruction. Each is distinguished by its own
special characteristics.
\end{abstract}
\tableofcontents
This is the first experience of \LaTeX.
\chapter{Aesop Fables}
\section{The \textsl{Ant} and the \textsl{Dove}}
\itshape
An ant went to the bank of a river to quench its thirst, and
being carried away by the rush of the stream, was on the
point of drowning.
\upshape
A \textsl{Dove}\marginpar{Pigeon, an emblem of peace.} sitting on a tree overhanging the water plucked a
leaf and let it fall into the stream close to her. The Ant
climbed onto it and floated in safety to the bank.
\section{The {\it Dog}\/ in the Manger}
A \textbf{\textit{dog}} lay in a manger, and by his growling and snapping
prevented the oxen from eating the hay which had been
placed for them.
‘‘What a selfish Dog!’’ said one of them to his companions;
‘‘he cannot eat the hay himself, and yet refuses to allow
those to eat who can.’’
\chapter{The \textsc{Eagle} and the Arrow}
An eagle sat on a lofty rock, watching the movements of a
Hare whom he sought to make his prey.
An archer, who saw the Eagle from a place of concealment,
took an accurate aim and wounded him mortally.
\end{document}
